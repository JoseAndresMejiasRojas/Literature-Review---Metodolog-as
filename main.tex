\documentclass[conference]{IEEEtran}
\IEEEoverridecommandlockouts
% The preceding line is only needed to identify funding in the first footnote. If that is unneeded, please comment it out.
\usepackage{cite}
\usepackage{amsmath,amssymb,amsfonts}
\usepackage{algorithmic}
\usepackage{graphicx}
\usepackage{textcomp}
\usepackage{xcolor}
\usepackage{comment}

\def\BibTeX{{\rm B\kern-.05em{\sc i\kern-.025em b}\kern-.08em
    T\kern-.1667em\lower.7ex\hbox{E}\kern-.125emX}}
\begin{document}

\title{Literature Review: Technology Platforms to Support Cancer Telerehabilitation}

\author{\IEEEauthorblockN{Jose Andrés Mejías Rojas}
\IEEEauthorblockA{\textit{Universidad de Costa Rica}\\San José, Costa Rica\\
jose.mejiasrojas@ucr.ac.cr}}

\maketitle

%\begin{abstract}
%This document is a model and instructions for \LaTeX.
%This and the IEEEtran.cls file define the components of your paper [title, text, heads, etc.]. *CRITICAL: Do Not Use Symbols, Special Characters, Footnotes, 
%or Math in Paper Title or Abstract.
%\end{abstract}

%\begin{IEEEkeywords}
%component, formatting, style, styling, insert
%\end{IEEEkeywords}

% Describa por qué el estudio es importante, qué lo diferencia de estudios existentes y cuál es el aporte que la revisión de literatura tendría para su posible tesis o TFIA.
\section{Introduction}
Cancer is one of the biggest diseases around the world. Cancer survivors need rehabilitation to improve their daily life and get back to normal. However, humanity encountered a terrible obstacle in 2020: COVID-19. Unfortunately, this pandemic has been forcing people to stay home as long as they can, wear masks on public places, and work from home. Is not that hard to imagine how complicated can be for cancer survivors regarding their rehabilitation. Therefore, telerehabilitation started to play a crucial role in this situation.

Telerehabilitation is rehabilitation at home. Patients follow a strict medical advice such as medication or therapy sessions. However, there is a clear limitation: remoteness, specially with elderly. There is a need to improve usability and maintainability for platforms who support telerehabilitation. If these kind of platforms can't offer an acceptable user experience, it may affect the patients' condition. 

This paper explains the protocol and results of a literature review to gather information related to telerehabilitation for cancer. The purpose of this review is to seek out needs concerning technology platforms that support this kind of rehabilitation. The difference between this study and other papers is we will look for the entire spectrum like technical decisions, design decision, validation, evaluations, among others, taking into account the pandemic. Hence, this paper will facilitate future work that can explore the specific aspects the review helps to pop them up.

% Antecedentes (existen trabajos similares---revisiones de literatura relacionadas)
% \section{Related Work}

\section{Systematic Literature Review}

\subsection{Planning the Review}

% Objetivos o preguntas a contestar
\subsubsection{Research Question}
% ¿Cómo se caracterizan las plataformas tecnológicas que apoyan la telerehabilitación de pacientes con cáncer?
The primary research question for the study is: What are the characteristics of technology platforms that support telerehabilitation of cancer patients? This primary question can be divide into four secondary questions: 1) How were the platforms evaluated? 2) What were the features offered by the platforms to improve the user experience? 3) What problems were reported? 4) What technical decisions were taken to implement the platform?

% Detalle la hilera de búsqueda genérica y la hilera de búsqueda detallada para cada fuente de información.
\subsubsection{Search Strategy}
Firstly, we made a trivial research to get keywords. Those keywords were used to build the query string. %PONER CITAS PARA INCLUIR eHealth

A preliminary search was made in some databases to determine if there is information related our topic and also to improve the query. After some iterations, we come out with the following query string: ("eHealth" OR "mHealth" OR "Mobile Health" OR “e-health” OR “m-health”) AND ("breast cancer") AND (telerehablitation OR rehabilitation OR rehab).

We decided to use Scopus, Springer Link, and ScienceDirect on account of the data we gathered with the query string.

% Tabla con los resultados de las búsquedas en cada base de datos

\subsubsection{Study Selection}
% Inclusión-Exclusión
The exclusion criteria we defined was based on language, year, and accessibility.  We decided to use English as the language due to the numbers of results on the sources after the execution of the query. Regarding the year, we defined a range between 2010 and 2021 because in 2010 the smartphones era started. Finally, we filtered the results to gather those with full access.

For the inclusion criteria, we defined that every paper needed to be a primary investigation and reviewed by peers. Also, the title, abstract, and keywords must include the following concepts or their synonyms: eHealth, rehabilitation, breast cancer, and evaluations.

% Pendiente definir el subtítulo
\subsubsection{Screening}
To start the review and gather relevant data we defined variables by research question. The first one, related to how the platforms were evaluated, we categorized it as experts review, questionnaires, heuristic evaluations, interview with users, users review, and rehab. The second one, related to the features to improve the user experience, we decided to define popular features like gamification, reminders, calendar integration, among others. The variables for the third question were UX, rehab, and technology. We needed to categorize them to detect common problems. The last one, related to technical decisions to implement the platform, we sought out for platforms like mobile apps, web applications, and technologies such as iOS, Android, development frameworks, and cloud services. 

\subsection{Conducting the Review}


\begin{comment}
%Comments

\end{comment}

\bibliography{references}
\bibliographystyle{IEEEtran}

\end{document}