\documentclass[conference]{IEEEtran}
\IEEEoverridecommandlockouts
% The preceding line is only needed to identify funding in the first footnote. If that is unneeded, please comment it out.
\usepackage{cite}
\usepackage{amsmath,amssymb,amsfonts}
\usepackage{algorithmic}
\usepackage{graphicx}
\usepackage{textcomp}
\usepackage{xcolor}
\usepackage{comment}

\def\BibTeX{{\rm B\kern-.05em{\sc i\kern-.025em b}\kern-.08em
    T\kern-.1667em\lower.7ex\hbox{E}\kern-.125emX}}
\begin{document}

\title{Guidelines to create usable CASE tools\\
}

\author{\IEEEauthorblockN{Jose Andrés Mejías Rojas}
\IEEEauthorblockA{\textit{Universidad de Costa Rica}\\San José, Costa Rica\\
jose.mejiasrojas@ucr.ac.cr}}

\maketitle

%\begin{abstract}
%This document is a model and instructions for \LaTeX.
%This and the IEEEtran.cls file define the components of your paper [title, text, heads, etc.]. *CRITICAL: Do Not Use Symbols, Special Characters, Footnotes, 
%or Math in Paper Title or Abstract.
%\end{abstract}

%\begin{IEEEkeywords}
%component, formatting, style, styling, insert
%\end{IEEEkeywords}

\section{Research intention}


Test \cite{helena_et_al}

%we constructed a set of guidelines for CASE practitioners in order to aid acceptance, use, and adoption  among the final users.

\section{Related work}

Test

%\section{Methdology}
\begin{comment}
%Comments

\end{comment}

\bibliography{references}
\bibliographystyle{IEEEtran}

\end{document}