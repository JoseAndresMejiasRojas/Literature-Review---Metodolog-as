\documentclass[conference]{IEEEtran}
\IEEEoverridecommandlockouts
% The preceding line is only needed to identify funding in the first footnote. If that is unneeded, please comment it out.
\usepackage{cite}
\usepackage{amsmath,amssymb,amsfonts}
\usepackage{algorithmic}
\usepackage{graphicx}
\usepackage{textcomp}
\usepackage{xcolor}
\usepackage{comment}

\def\BibTeX{{\rm B\kern-.05em{\sc i\kern-.025em b}\kern-.08em
    T\kern-.1667em\lower.7ex\hbox{E}\kern-.125emX}}
\begin{document}

\title{Literature Review: Technology Platforms to Support Cancer Telerehabilitation}

\author{\IEEEauthorblockN{Jose Andrés Mejías Rojas}
\IEEEauthorblockA{\textit{Universidad de Costa Rica}\\San José, Costa Rica\\
jose.mejiasrojas@ucr.ac.cr}}

\maketitle

%\begin{abstract}
%This document is a model and instructions for \LaTeX.
%This and the IEEEtran.cls file define the components of your paper [title, text, heads, etc.]. *CRITICAL: Do Not Use Symbols, Special Characters, Footnotes, 
%or Math in Paper Title or Abstract.
%\end{abstract}

%\begin{IEEEkeywords}
%component, formatting, style, styling, insert
%\end{IEEEkeywords}

% Describa por qué el estudio es importante, qué lo diferencia de estudios existentes y cuál es el aporte que la revisión de literatura tendría para su posible tesis o TFIA.
\section{Introduction}
Humanity has been facing a war against cancer over the years. People already know how terrible the consequences are. Furthermore, there is a specific cancer that women in particular have been struggling: breast cancer. The battle against breast cancer has been complicated and, unfortunately, mortal \cite{burbank_knowledge_2006}. Cancer survivors need to decrease the side effects of surgery, so there is a need of a good rehabilitation to improve their daily life and get back to normal \cite{de_rezende_telerehabilitation_2021}. However, humanity encountered an obstacle in 2020: COVID-19. Unfortunately, this pandemic has been forcing people to stay home as long as they can, wear masks on public places, and work from home. Is not that hard to imagine how complicated can be for cancer survivors regarding their rehabilitation.

Telerehabilitation started to play a crucial role in this situation. Telerehabilitation is a subspecialty of telehealth that uses technologies to provide treatments \cite{galiano-castillo_agreement_2014, van_der_linden_feasibility_2018}. Patients follow a strict medical advice such as medication or therapy sessions.  

This paper explains the protocol and results of a literature review to gather information related to telerehabilitation of breast cancer. The purpose of this review is to seek out needs concerning technology platforms that support this subspecialty. The difference between this study and other papers is that we looked for the entire spectrum like how the platforms were evaluated, what were the main features to improve usability, what problems the platforms faced, and what were the crucial decisions to implement the respective platform. Hence, this paper will facilitate future work that may explore specific aspects the review helps to pop up.

% Antecedentes (existen trabajos similares---revisiones de literatura relacionadas)
% \section{Related Work}

\section{Review}

\subsection{Planning the Review}

% Objetivos o preguntas a contestar
\subsubsection{Research Question}
% ¿Cómo se caracterizan las plataformas tecnológicas que apoyan la telerehabilitación de pacientes con cáncer?
The primary research question for the study was: What are the characteristics of technology platforms that support telerehabilitation of cancer patients? This primary question was divided into four secondary questions: 1) How were the platforms evaluated? 2) What were the features offered by the platforms to improve the user experience? 3) What problems were reported? 4) What technical decisions were taken to implement the platform?

% Detalle la hilera de búsqueda genérica y la hilera de búsqueda detallada para cada fuente de información.
\subsubsection{Search Strategy}
Firstly, we made a trivial research to get keywords. Those keywords were used to build the query string. We checked papers related to telerehabilitation and we got remarkable terms like eHealth, and mHealth \cite{iacono_scoping_2016}. Also, we find out that the term `adherence' is used to talk about how well a patient follows a treatment or therapy \cite{JEMINIWA201959}.

Based on the collected keywords mentioned before, a preliminary search was made in some databases to determine how much information were around and to improve the query. After some iterations, we came out with the following query string: (``eHealth'' OR ``mHealth'' OR ``Mobile Health'' OR ``e-health'' OR ``m-health'') AND (``breast cancer'') AND (``telerehablitation'' OR ``rehabilitation'' OR ``rehab'').

We decided to use Scopus, Springer Link, and ScienceDirect on account of the data we gathered with the query string. We didn't define specific queries for each database. Table \ref{table:preliminary_results} shows the results using the query string in the databases we defined without any exclusion criteria.

\begin{table}[h!]
\centering
\begin{tabular}{||c c c||} 
 \hline
 Scopus & Springer Link & ScienceDirect \\
 \hline\hline
 1550 & 547 & 248 \\ 
 \hline
\end{tabular} \\ [1ex] 
\caption{Preliminary results without the exclusion criteria}
\label{table:preliminary_results}
\end{table}

\subsubsection{Study Selection}
% Inclusión-Exclusión
The exclusion criteria we determined was based on language, year, and accessibility.  We decided to use English. According to the International Telecommunication Union, in 2015 around the world 78.3 per 100 inhabitants were covered by at least a 3G mobile network \cite{ITU}. Therefore we defined a range between 2015 and 2021. Finally, we filtered the results to gather those with full access.

For the inclusion criteria, we defined that every paper needed to be a primary investigation and reviewed by peers. Also, the title, abstract, and keywords must include the following concepts or their synonyms: eHealth, rehabilitation, breast cancer, and evaluations.

% Pendiente definir el subtítulo
\subsubsection{Data Extraction}
We defined variables by research question to extract relevant data. The first one, related to how the platforms were evaluated, we categorized it as experts review, questionnaires, heuristic evaluations, interview with users, users review, and rehab. The second one, related to the features to improve the user experience, we decided to define popular features like gamification, reminders, calendar integration, among others. The variables for the third question were UX, rehab, and technology. We needed to categorize them to detect common problems. The last one, related to technical decisions to implement the platform, we sought out for platforms like mobile apps, web applications, and technologies such as iOS, Android, development frameworks, and cloud services. 

\subsection{Conducting the Review}


\begin{comment}
%Comments

\end{comment}

\bibliography{references}
\bibliographystyle{IEEEtran}

\end{document}